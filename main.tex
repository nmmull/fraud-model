\documentclass{article}[10pt]

% palatino
\usepackage{mathpazo}

% 1 inch margins
\usepackage[margin=1in]{geometry}

% line spacing
%% \usepackage{setspace}
%% \setstretch{1.15}

% better quotes
\usepackage{csquotes}

% standard math tools
\usepackage{amsthm, amssymb, mathtools, bussproofs}

% highlights
\usepackage{soul}

\newcommand{\X}{\mathcal X}
\newcommand{\F}{\mathcal F}
\newcommand{\cL}{\mathcal L}
\newcommand{\zo}{\{0, 1\}}
\newcommand{\ow}{\text{otherwise}}


%% \setlength{\parindent}{0cm}
%% \setlength{\parskip}{1cm}

\title{A Theoretical Model for Fraud}
\author{James Le and Nathan Mull}
\begin{document}
\maketitle

\section{The Model}

We begin by describing an abstract model of fraud detection, along the lines of what is done by Paladini et.\ al \cite{paladini2023advancing} and du Preez et.\ al \cite{du2024systems}.
Fix a set $\X$ representing possible \textbf{transactions} with a designated unit element $\bullet$.
The unit element represents \enquote{no transaction.}
Also fix a \textbf{cost function} of the form
\begin{align*}
\mu : \X \to \mathbb R^{\geq 0}
\end{align*}
where $\mu(\bullet) = 0$.
We think of the cost function as describing the amount of a given transaction, so no transaction has zero cost.
A \textbf{fraudster} is a function of the following form.
\begin{align*}
  \F : (\X \times \zo \times \zo)^* \to \X
\end{align*}
The input to a fraudster is called transcript (defined formally below), and is of the form $\langle Y_i, f_i, l_i \rangle_{i = 1}^n$ where $Y_i$ is the transaction made at time step $i$, the value $f_i$ is an indicator of whether or not $Y_i$ was a fraudulent transaction, and $l_i$ is the label given to the transaction by the interacting fraud detecter (defined below).
We think of a fraudster as determining what transaction to make given the previous transactions made by their target or themselves.
In particular, the fraudster has access to whether or not they were successful in their previous fraudulent transactions.

A fraudster interacts with a \textbf{detecter}, which is a function of the following form.
\begin{align*}
  \cL : (\X \times \zo \times \zo)^* \to \zo^{\X}
\end{align*}
We will typically assume that $\cL(\cdot)(\bullet) = 0$ since \enquote{no transaction} should never be labeled as fraudulent.
We think of the detecter as determining how it will decide its next label according to what transaction it sees.

Consider a sequence of transactions $\langle X_i \rangle_{i = 1}^n$.
A $(\cL, \F)$-\textbf{transcript} for $\langle X_i \rangle_{i = 1}^n$ is a sequence of the from $\langle  Y_i, f_i, l_i \rangle_{i = 1}^n$ from $(\X \times \zo \times \zo)^*$ defined inductively as follows.
\begin{align*}
  Y_{k + 1} &=
  \begin{cases}
    X_{k + 1} & \F(\langle Y_i, f_i, l_i\rangle_{i = 1}^k) = \bullet \\
    \F(\langle Y_i, f_i, l_i \rangle_{i = 1}^k) & \ow
  \end{cases} \\
  f_{k + 1} &=
  \begin{cases}
    0 & \F(\langle Y_i, f_i, l_i\rangle_{i = 1}^k) = \bullet \\
    1 & \ow
  \end{cases} \\
  l_{k + 1} &= \cL(\langle Y_i, f_i, l_i\rangle_{i = 1}^k)(Y_{k + 1})
\end{align*}
In other words, the fraudster has the ability to hijack transactions, and the detecter attempts to determine at each time step if the transaction has been hijacked.
The \textbf{false negative loss} of a detecter on a transcript $\langle Y_i, f_i, l_i \rangle_{i = 1}^k$ is define as follows.
\begin{align*}
  \ell_{\text{FN}}(\langle Y_i, f_i, l_i \rangle_{i = 1}^k) = \sum_{i = 1}^k \mu(Y_i) \mathbb 1[l_i = 0 \land f_i = 1]
\end{align*}
The \textbf{false positive loss} is defined similarly, relative to a constant $\gamma_{\text{FP}}$.
\begin{align*}
  \ell_{\text{FP}}(\langle Y_i, f_i, l_i \rangle_{i = 1}^k) = \sum_{i = 1}^k \gamma_{\text{FP}} \mathbb 1[l_i = 1 \land f_i = 0]
\end{align*}
The constant represents the penalty for labeling, say, a bank users transaction as fraudulent, and should large enough dissuade the detecter from labeling all transactions as fraudulent.
Finally, the \textbf{total loss} is the sum of the false negative loss and the false positive loss.
\begin{align*}
  \ell(\mathcal T) = \ell_{\text{FN}}(\mathcal T) + \ell_{\text{FP}}(\mathcal T)
\end{align*}
This model obviously does not capture all facets of the fraud detection problem, but it provides a starting point for being able to state theoretical results.

There are a couple features about the model to note.
\begin{itemize}
\item
  The fraudster and the detecter determine their parts of a transaction simultaneously, so this model meshes naturally with game-theoretic ones.
  In particular, it is possible to construct a (very large) game matrix in which the fraudster chooses a transaction, the detecter choose a binary function on transactions, and the entry in the matrix is the total loss on the singleton transcript for this transaction.
Furthermore, the size of the domain for the detecter can be drastically decreased by considering a fixed set of expert detecters.
\item
  The model is too general too say anything all that interesting.
\end{itemize}



\bibliographystyle{plain}
\bibliography{refs}
\end{document}
